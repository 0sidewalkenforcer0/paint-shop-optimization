% +------------------------------------------------------------------------------+ %
% | EDIT THESE SETTINGS ACCORDING TO YOUR THESIS                                 | %
% +------------------------------------------------------------------------------+ %

\newcommand{\practical}{Quantum Computing Programmierung} % comment for master seminar
%\newcommand{\seminar}{Vertiefte Themen in Mobilen und Verteilten Systemen} % uncomment for master seminar

\newcommand{\authorA}{Alexander Lankheit}
\newcommand{\authorB}{Sarah Gerner}
\newcommand{\authorC}{Verena Jones}
\newcommand{\authorD}{Jingcheng Wu}

\newcommand{\supervisor}{Mallory Miller}

\newcommand{\thesistitle}{Projektarbeit am Lehrstuhl für mobile und verteilte Systeme}

\newcommand{\thesisabstract}{% \/ put your thesis abstract below \/
Diese Arbeit beschreibt einen Lösungsansatz bezüglich des Multi Car Multi Color Paintshop Problem unter der Verwendung von Quanten Computing im Rahmen des Projektes QC-Challenge am Lehrstuhl für Mobile und Verteilte Systeme..  
}% <-- mind this closing brace!

\newcommand{\thesisauthorship}{% \/ describe who wrote what below \/

{\authorC} hat die Abschnitte ~\ref{sec:intro} verfasst.
%Alice Jones hat die Abschnitte~\ref{sec:intro}, \ref{sec:related} und~\ref{subsec:qubo} verfasst. Bob Smith hat die Abschnitte~\ref{subsec:training} und \ref{subsec:qgm} verfasst. Den Abschnitt~\ref{sec:conclusion} haben beide Autoren gemeinsam verfasst. (In einer wissenschaftlichen Arbeit ohne Prüfungszweck stünde hier eine Danksagung!)
}% <-- mind this closing brace!


\selectlanguage{ngerman} %options: english, ngerman
